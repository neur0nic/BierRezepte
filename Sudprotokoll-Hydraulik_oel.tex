\documentclass[12pt,oneside,a4paper]{scrartcl}
%Einstellungen der Seitenraender
\usepackage[left=1.5cm,right=1.5cm,top=2cm,bottom=2.5cm]{geometry}

\usepackage{ngerman}
\usepackage[utf8]{inputenc}
\usepackage{setspace}
\usepackage[pdftex]{graphicx}

%Zum Erstellen von Diagrammen
\usepackage{tikz}
\usepackage{pgfplots}

\usepackage{bigstrut}

%Hyperlinks innerhalb des PDF Dokuments
\usepackage[bookmarks,
ngerman,
pageanchor,
hyperindex,
hidelinks,
pdffitwindow,
pdftitle={Sudprotokoll: Pale Ale / Hydraulik-\O{}l},
pdfauthor={Stephan Mertens},
]{hyperref}
\usepackage{bookmark}

%um text einzufaerben
\usepackage{color}

%tabular zeug
\usepackage{dcolumn}

% Fuer zusaaetzliche Zeichen
\usepackage{textcomp}
\usepackage{wasysym}
\usepackage{marvosym}

\begin{document}
	%Header
	\begin{minipage}[c]{0.70\textwidth}
		\section*{\hspace{-.4cm}Sudprotokoll: Pale Ale / Hydraulik-\O{}l}
	\end{minipage}
	\begin{minipage}[c]{0.29\textwidth}
		\subsection*{am 26. August 2016}
	\end{minipage}
	\rule{\textwidth}{1pt}\vspace{5mm}
	%
	\subsection*{Zutaten}
	%
	%Schuettung
	\paragraph{Schüttung:}
	\begin{tabular}[t]{m{8cm} m{2cm} m{1cm}}
		Pale Ale Malt & 0,6 & kg \bigstrut\\
		Münchner Malz & 1,2 & kg \bigstrut\\\hline
		Gesamtschüttung & 1,8 & kg\bigstrut
	\end{tabular}\\
	
	\vspace{.25cm}
	\hspace{1cm}Angestrebte Stammwürze: 11,5\,°P
	%
	%Hopfung
	\paragraph{Hopfung:}
	\begin{tabular}[t]{m{2.5cm} m{5cm} m{0.5cm} m{1cm} m{0.5cm} m{1cm}}
		Bitterhopen: & Magnum & 9 & g &  & 12,2\,\%$\alpha$ \\%28,6
		Aromahopfen: & Simcoe & 15 & g &  & 12,9\,\%$\alpha$ \\%8,6
		Dryhopping: & Simcoe & 30 & g & & 12,9\,\%$\alpha$
	\end{tabular}\\
	
	\vspace{.25cm}
	\hspace{1cm}Errechnete Bittere: 37,5\,IBU
	%
	%Hefe
	\paragraph{Hefe:}
	Safale US-05
	
	\onehalfspacing{
		%
		%Sudverlauf
		\subsection*{Sudverlauf - Infusionsverfahren (hoch-kurz)}	
		%
		%Maischen
		\paragraph{Maischen:}
		\begin{tabbing}\hspace{1cm} \=
			\hspace{1cm} \= \hspace{1cm} \=\hspace{1cm} \=\hspace{1cm} \=\hspace{1cm} \= \hspace{1cm} \= \hspace{1cm} \= \hspace{1cm} \= \hspace{1cm} \= \kill
			\> 7\,l vorlegen bei 64\,°C\\
			\> \textit{Maltose-/$\beta$-Amylaserast} für 45 Minuten von 20:15 -- 21:02 Uhr.\\
			\> \> Aufheizen auf 72\,°C.\\
			\> \textit{Verzuckerungs-/$\alpha$-Amylaserast} bis zur Verzuckerung von 21:09 -- 21:50 Uhr.\\
			\> \> \> Jodprobe: \> \> \Square \ positiv \> \> \CheckedBox \ negativ\\
			\> \> Aufheizen auf 78\,°C.\\\\
			%
			\> Kommentar: \>\>\>Direkt beim Einmaischen überhitzt auf 72\,°C. Ein paar Eiswürfel reinge-\\
			\>\>\>\>worfen und dann drauf geschissen.
		\end{tabbing}
		%
		%Läutern
		\paragraph{Läutern:} von 22:02 -- 23:10 Uhr.
		\begin{tabbing}
			\hspace{1cm} \= \hspace{1cm} \= \hspace{1cm} \= \hspace{1cm} \= \hspace{1cm} \=\hspace{1cm} \=\hspace{1cm} \=\hspace{1cm} \= \kill
			\> 1l Wasser vorlegen.\\
			\> \> 1. Nachguss: \> \> 6 l\\
			\> \> 2. Nachguss: \> \> 5 l\\
			\> \> \> Aufhacken: \> \> \CheckedBox \ ja \> \> \Square \ nein\\
			\> Würze: \> \> \> 5,5\,°P \> bei \>64\,°C \> $\Rightarrow$ \> 10,6\,°P\\
		\end{tabbing}
		%
		%Würzekochung
		\paragraph{Würzekochung:} für 80 Minuten von 23:20 -- 0:40 Uhr.\\
		%Whirpool
		\paragraph{Whirpoolrast:} von 0:40 -- 1:00 Uhr.
		\begin{tabbing}
			\hspace{1cm} \= \hspace{1cm} \= \hspace{1cm} \= \hspace{1cm} \= \hspace{1cm} \= \hspace{1cm} \= \hspace{1cm} \= \hspace{1cm} \= \kill
			\> Kommentar: \>\>\> Hat sich nicht so schön geklärt. Trubkegel trotzdem schön und Fest. NT.\\
			\> \> Ausschlagwürze: \> \> \> \> \> ca. 7,25\,°P bei 67\,°C $\Rightarrow$ 13,3\,°P\\
			\> \> Verdünnen: \> \> \> \> \> ($8l \cdot$$ 13,3^\circ P)/$($11,5^\circ P)= 9,3l$\\
			\> \> Sudhausausbeute: \> \> \> \> \>  60\,\% mit 9,3\,l\\
			\> \> Geschätzter Alkoholgehalt: \> \> \> \> \> 5,2\,\% Vol.
		\end{tabbing}
		%
		%Hefegabe
		\subsection*{Gärung}
		\paragraph{Hefegabe:} und $O_2$-Gabe um 9:20 Uhr.
		\paragraph{Dryhopping:} 30\,g Simcoe für 7 Tage von .
		%
		\paragraph{Gärverlauf:}
		\begin{tabbing}
			\hspace{1cm} \= \hspace{1cm} \= \hspace{1cm} \= \hspace{1cm} \= \hspace{1cm} \= \hspace{1cm} \= \hspace{1cm} \= \hspace{1cm} \= \kill
			\> Angestellt am Samstag, den 27.08. um 9:20 Uhr.\\
			\> \> ungefähre Gärtemperatur  21\,°C.\\
			\> Erste Kräusen am Samstag,  Tag Nr. 0.\\
			\> \> ungefähre Gärtemperatur 26\,°C.\\
			\> Kräusen fallen zusammen am Samstag, Tag Nr. 0.\\
			\> \> ungefähre Gärtemperatur 26\,°C.\\
			\> \> 2 Runden Waschmaschine und dannach gabs keine Kräusen mehr\\
			\> Gärung beendet am ...................., Tag Nr. .......... .\\
			\> \> ungefähre Gärtemperatur ..........\,°C.\\
			\> Hopfengabe am ...................., Tag Nr. .......... .\\
			\> Hopfen entfernen am ...................., Tag Nr. .......... .
		\end{tabbing}
		%
		\subsection*{Schlauchen}
		\begin{tabbing}
			\hspace{1cm} \= \hspace{1cm} \= \hspace{1cm} \= \hspace{1cm} \= \hspace{1cm} \= \hspace{1cm} \= \hspace{1cm} \= \hspace{1cm} \= \kill
			\> ...................., den  .................... um .......... Uhr.\\
			\> Zuckergabe: \> \> \> \> ..........g/l\\
			\> Gesamtabfüllung: \> \> \> \> .......... 0,5\,l Flasche\\
			\> \> \> \> \> .......... 0,75\,l Flasche\\
			\\
			\> Kommentar: \>\>\>\rule[-0.2cm]{13cm}{1pt}\\
			\> \>  \rule[-0.2cm]{15.3cm}{1pt}\\
			\> \>  \rule[-0.2cm]{15.3cm}{1pt}\\
			\> \>  \rule[-0.2cm]{15.3cm}{1pt}\\
			\> \>  \rule[-0.2cm]{15.3cm}{1pt}\\
			\> \>  \rule[-0.2cm]{15.3cm}{1pt}\\
			\> \>  \rule[-0.2cm]{15.3cm}{1pt}			
		\end{tabbing}
	\end{document}