\documentclass[12pt,oneside,a4paper]{scrartcl}
%Einstellungen der Seitenraender
\usepackage[left=1.5cm,right=1.5cm,top=2cm,bottom=2.5cm]{geometry}

\usepackage{ngerman}
\usepackage[utf8]{inputenc}
\usepackage{setspace}
\usepackage[pdftex]{graphicx}

%Zum Erstellen von Diagrammen
\usepackage{tikz}
\usepackage{pgfplots}

\usepackage{bigstrut}

%Hyperlinks innerhalb des PDF Dokuments
\usepackage[bookmarks,
ngerman,
pageanchor,
hyperindex,
hidelinks,
pdffitwindow,
pdftitle={Sudprotokoll: Pale Ale / Hydraulik-\O{}l},
pdfauthor={Stephan Mertens},
]{hyperref}
\usepackage{bookmark}

%um text einzufaerben
\usepackage{color}

%tabular zeug
\usepackage{dcolumn}

% Fuer zusaaetzliche Zeichen
\usepackage{textcomp}
\usepackage{wasysym}
\usepackage{marvosym}

\begin{document}
%Header
	\begin{minipage}[c]{0.70\textwidth}
		\section*{\hspace{-.4cm}Sudprotokoll: Pale Ale / Hydraulik-\O{}l}
	\end{minipage}
	\begin{minipage}[c]{0.29\textwidth}
		\subsection*{am 26. August 2016}
	\end{minipage}
	\rule{\textwidth}{1pt}
%
\subsection*{Zutaten}
%
%Schuettung
\paragraph{Schüttung:}
	\begin{tabular}[t]{m{8cm} m{2cm} m{1cm}}
		Pale Ale Malt & 0,5 & kg \bigstrut\\
		Münchner Malz & 1 & kg \bigstrut\\\hline
		Gesamtschüttung & 1,5 & kg\bigstrut
	\end{tabular}\\

\vspace{.25cm}
\hspace{1cm}Angestrebte Stammwürze: 11,5 °P
%
%Hopfung
\paragraph{Hopfung:}
	\begin{tabular}[t]{m{2.5cm} m{5cm} m{0.5cm} m{1cm} m{0.5cm} m{1cm}}
		Bitterhopen: & Magnum & 8 & g &  & 15\%$\alpha$ \\
		Aromahopfen: & Simcoe & 15 & g &  &\%$\alpha$
	\end{tabular}\\

\vspace{.25cm}
\hspace{1cm}Errechnete Bittere:  IBU
%
%Hefe
\paragraph{Hefe:}
	Safale US-05

\onehalfspacing{
%
%Sudverlauf
\subsection*{Sudverlauf - Infusionsverfahren}	
%
%Maischen
\paragraph{Maischen:}
	\begin{tabbing}\hspace{1cm} \=
		\hspace{1cm} \= \hspace{1cm} \=\hspace{1cm} \=\hspace{1cm} \=\hspace{1cm} \= \hspace{1cm} \= \hspace{1cm} \= \hspace{1cm} \= \hspace{1cm} \= \kill
		\> 7l vorlegen bei 64$^\circ C$\\
		\> \textit{Maltose-/$\beta$-Amylaserast} für 45 Minuten von  Uhr.\\
		\> \> Aufheizen auf 72$^\circ C$.\\
		\> \textit{Verzuckerungs-/$\alpha$-Amylaserast} bis zur Verzuckerung von  Uhr.\\
		\> \> \> Jodprobe: \> \> \Square \ positiv \> \> \CheckedBox \ negativ\\
		\> \> Aufheizen auf 78$^\circ C$.\\\\
		%
		\> Kommentar: \>\>\>\\
	\end{tabbing}
%
%Läutern
\paragraph{Läutern:} von  Uhr.
	\begin{tabbing}
		\hspace{1cm} \= \hspace{1cm} \= \hspace{1cm} \= \hspace{1cm} \= \hspace{1cm} \=\hspace{1cm} \=\hspace{1cm} \=\hspace{1cm} \= \kill
		\> 1l Wasser vorlegen.\\
		\> \> 1. Nachguss: \> \> 7 l\\
		\> \> 2. Nachguss: \> \> 5 l\\
		\> \> \> Aufhacken: \> \> \CheckedBox \ ja \> \> \Square \ nein\\
		%\> \> 3. Nachguss: \> \> 2 l\\
		\> Würze: \> \> \> $^\circ P$ \> bei \>$^\circ C$ \> $\Rightarrow$ \> $^\circ P$\\
		\> Glattwasser: \> \> \>$^\circ P$ \> bei \> $^\circ C$ \> $\Rightarrow$ \> 6,5$^\circ P$\\\\
		\> Kommentar: \>\>\> \\
	\end{tabbing}
%
%Würzekochung
\paragraph{Würzekochung:} für 80 Minuten von  Uhr.\\
%Whirpool
\paragraph{Whirpoolrast:} von  Uhr.
	\begin{tabbing}
		\hspace{1cm} \= \hspace{1cm} \= \hspace{1cm} \= \hspace{1cm} \= \hspace{1cm} \= \hspace{1cm} \= \hspace{1cm} \= \hspace{1cm} \= \kill
		\> Kommentar: \>\>\> \\
		\> \> \> ca. $^\circ P$ \> \> bei \> $^\circ C$ \> $\Rightarrow$ \> $^\circ P$\\
		\> \> Verdünnen: \> \> \> \> \> ($l \cdot$$ ^\circ P)/$($^\circ P)= l$\\
		\> \> Sudhausausbeute: \> \> \> \> \>  \% mit l\\
		\> \> Geschätzter Alkoholgehalt: \> \> \> \> \> \% Vol.
	\end{tabbing}
%
%Hefegabe
\paragraph{Hefegabe} und $O_2$-Gabe um  Uhr.
%
\subsection*{Gärverlauf}
	\begin{tabbing}
		\hspace{1cm} \= \hspace{1cm} \= \hspace{1cm} \= \hspace{1cm} \= \hspace{1cm} \= \hspace{1cm} \= \hspace{1cm} \= \hspace{1cm} \= \kill
		\> Angestellt am Samstag, den  um  Uhr.\\
		\> \> ungefähre Gärtemperatur  $^\circ C$.\\
		\> Erste Kräusen am Samstag,  Tag Nr. .\\
		\> \> ungefähre Gärtemperatur $^\circ C$.\\
		\> Kräusen fallen zusammen am Mittwoch, Tag Nr. .\\
		\> \> ungefähre Gärtemperatur $^\circ C$.\\
		\> Gärung beendet am Freitag, Tag Nr. .\\
		\> \> ungefähre Gärtemperatur $^\circ C$.
	\end{tabbing}
%
\subsection*{Schlauchen}
	\begin{tabbing}
		\hspace{1cm} \= \hspace{1cm} \= \hspace{1cm} \= \hspace{1cm} \= \hspace{1cm} \= \hspace{1cm} \= \hspace{1cm} \= \hspace{1cm} \= \kill
		\> Samstag, den  um  Uhr.\\
		\> Zuckergabe:  g/l\\
		\> Gesamtabfüllung: \> \> \> \>  \> 0,5l Flasche\\
		\> \> \> \> \>  \> 0,75l Flasche\\
		\\
		\> Kommentar: \>\>\>\\
	\end{tabbing}
%
\subsection*{Verkostet}
\begin{tabbing}
	\hspace{1cm} \= \hspace{1cm} \= \hspace{1cm} \= \hspace{1cm} \= \hspace{1cm} \= \hspace{1cm} \= \hspace{1cm} \= \hspace{1cm} \= \kill
	\> Kommentar: \>\>\> \\
	\> \>  \rule[-0.2cm]{15.3cm}{1pt}\\
	\> \>  \rule[-0.2cm]{15.3cm}{1pt}\\
	\> \>  \rule[-0.2cm]{15.3cm}{1pt}\\		
	\> \>  \rule[-0.2cm]{15.3cm}{1pt}\\
	\> \>  \rule[-0.2cm]{15.3cm}{1pt}\\
	\> \>  \rule[-0.2cm]{15.3cm}{1pt}\\
	\> \>  \rule[-0.2cm]{15.3cm}{1pt}\\
	\> \>  \rule[-0.2cm]{15.3cm}{1pt}\\
	\> \>  \rule[-0.2cm]{15.3cm}{1pt}\\
	\> \>  \rule[-0.2cm]{15.3cm}{1pt}\\
	\> \>  \rule[-0.2cm]{15.3cm}{1pt}\\
	\> \>  \rule[-0.2cm]{15.3cm}{1pt}\\
	\> \>  \rule[-0.2cm]{15.3cm}{1pt}\\
	\> \>  \rule[-0.2cm]{15.3cm}{1pt}\\
	\> \>  \rule[-0.2cm]{15.3cm}{1pt}\\
	\> \>  \rule[-0.2cm]{15.3cm}{1pt}\\
	\> \>  \rule[-0.2cm]{15.3cm}{1pt}\\
	\> \>  \rule[-0.2cm]{15.3cm}{1pt}\\
	\> \>  \rule[-0.2cm]{15.3cm}{1pt}
\end{tabbing}}
\end{document}
