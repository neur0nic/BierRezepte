\documentclass[12pt,oneside,a4paper]{scrartcl}
%Einstellungen der Seitenraender
\usepackage[left=1.5cm,right=1.5cm,top=2cm,bottom=2.5cm]{geometry}

\usepackage{ngerman}
\usepackage[utf8]{inputenc}
\usepackage{setspace}
\usepackage[pdftex]{graphicx}
\pagenumbering{gobble}

%Zum Erstellen von Diagrammen
\usepackage{tikz}
\usepackage{pgfplots}

\usepackage{bigstrut}

%Hyperlinks innerhalb des PDF Dokuments
\usepackage[bookmarks,
ngerman,
pageanchor,
hyperindex,
hidelinks,
pdffitwindow,
pdftitle={Sudprotokoll: Dunkler Doppelbock},
pdfauthor={Stephan Mertens},
]{hyperref}
\usepackage{bookmark}

%um text einzufaerben
\usepackage{color}

%tabular zeug
\usepackage{dcolumn}

% Fuer zusaaetzliche Zeichen
\usepackage{textcomp}
\usepackage{wasysym}
\usepackage{marvosym}
\usepackage{siunitx}
\sisetup{locale = DE,
	per-mode=symbol-or-fraction}
\DeclareSIUnit\plato{^\circ P}
\DeclareSIUnit\peralpha{\%\,\alpha}
\DeclareSIUnit\ibu{IBU}
\DeclareSIUnit\pervol{\%\,Vol}
\newcommand{\Zeitluecke}{\rule[-0.1cm]{1.8cm}{0.5pt}}

\begin{document}
%Header
	\begin{minipage}[c]{0.70\textwidth}
		\section*{\hspace{-.4cm}Sudprotokoll: Dunkler Doppelbock}
	\end{minipage}
	\begin{minipage}[c]{0.29\textwidth}
		\subsection*{am }
	\end{minipage}
	\rule{\textwidth}{2pt}
%
\subsection*{Zutaten}
%
%Schuettung
\paragraph{Schüttung:}
	\begin{tabular}[t]{m{8cm} m{2cm} m{1cm}}
		Münchner Malz & \num{3,0} & \si{\kilogram} \bigstrut\\
		Cara Münch & \num{0,4} & \si{\kilogram} \bigstrut\\
		Sauermalz & \num{0,1} & \si{\kilogram} \bigstrut\\
		Haferflocken & \num{0,1} & \si{\kilogram} \bigstrut\\\hline\hline
		Gesamtschüttung & \num{3,6} & \si{\kilogram} \bigstrut
	\end{tabular}\\

\vspace{.25cm}
\hspace{1cm}Angestrebte Stammwürze: \SI{18}{\plato}
%
%Hopfung
\paragraph{Hopfung:}
	\begin{tabular}[t]{m{2.5cm} m{5cm} m{0.5cm} m{1cm} m{0.5cm} m{1cm}}
		Bitterhopen: & Hallertauer Tradition & \num{15} & \si{\gram} & \num{6,6} & \si{\peralpha} \\
		Aromahopfen: & Tettnanger &\num{15} & \si{\gram} & \num{3,3} &  \si{\peralpha}
	\end{tabular}\\

\vspace{.25cm}
\hspace{1cm}Errechnete Bittere: \SI{20,2}{\ibu}
%
%Hefe
\paragraph{Hefe:}
	Brewferm Lager
\vspace{1cm}
\onehalfspacing{
%
%Sudverlauf
\subsection*{Sudverlauf - Kombinationsverfahren}	
%
%Maischen
\paragraph{Maischen:}
	\begin{tabbing}\hspace{1cm} \=
		\hspace{1cm} \= \hspace{1cm} \=\hspace{1cm} \=\hspace{1cm} \=\hspace{1cm} \= \hspace{1cm} \= \hspace{1cm} \= \hspace{1cm} \= \hspace{1cm} \= \kill
		\> \SI{8}{\litre} vorlegen bei \SI{52}{\degreeCelsius}.\\
		%\> \> Aufheizen auf 35$^ \circ C$.\\
		%\> \textit{Gummi/Glucanaserast} für Minuten von - Uhr.\\
		%\> \> Aufheizen auf 44$^\circ C$.\\
		%\> \textit{Ferulasäurerast} für Minuten von - Uhr.\\
		%\> \> Aufheizen auf 52$^\circ C$.\\
		\> \textit{Eiweißrast} für 15 Minuten von \Zeitluecke -- \Zeitluecke Uhr.\\
		\> \> Aufheizen auf \SI{64}{\degreeCelsius}\\
		\> \textit{Maltose-/$\beta$-Amylaserast} für 45 Minuten von \Zeitluecke -- \Zeitluecke Uhr.\\
		\> \> Aufheizen auf \SI{72}{\degreeCelsius} mit Dekoktion.\\
		\> \> Volumen Dichmaische: $ V_{tot} \cdot \frac{\vartheta_2 - \vartheta_1}{\vartheta_b - \vartheta_1} = \rule[-0.1cm]{1cm}{0.5pt}\si{\litre} \cdot \frac{\SI{72}{\degreeCelsius} - \SI{64}{\degreeCelsius}}{\SI{100}{\degreeCelsius} - \SI{64}{\degreeCelsius}} = \num{0,22} \cdot V_{tot} = \rule[-0.1cm]{1cm}{0.5pt}\si{\litre}$\\
		\> \> Dickmaische für 15 Minuten kochen von \Zeitluecke -- \Zeitluecke Uhr.\\
		\> \textit{Verzuckerungs-/$\alpha$-Amylaserast} bis zur Verzuckerung von \Zeitluecke -- \Zeitluecke Uhr.\\
%		\> \> Zugabe der zweiten Schüttung.\\
		\> \> \> Jodprobe: \> \> \Square \ positiv \> \> \Square \ negativ\\
		\> \> Aufheizen auf \SI{78}{\degreeCelsius}.\\
		\> Abmaischen um \Zeitluecke Uhr.\\\\
		%
		\> Kommentar: \>\>\> \rule[-0.1cm]{13cm}{0.5pt}\\
			\> \>  \rule[-0.1cm]{15.3cm}{0.5pt}\\
			\> \>  \rule[-0.1cm]{15.3cm}{0.5pt}\\
			\> \>  \rule[-0.1cm]{15.3cm}{0.5pt}	
	\end{tabbing}
%
%Läutern
\pagebreak
\paragraph{Läutern:} von \Zeitluecke -- \Zeitluecke Uhr.
	\begin{tabbing}
		\hspace{1cm} \= \hspace{1cm} \= \hspace{1cm} \= \hspace{1cm} \= \hspace{1cm} \=\hspace{1cm} \=\hspace{1cm} \=\hspace{1cm} \= \kill
		\> 1l Wasser vorlegen.\\
		\> \> 1. Nachguss: \> \> \> \SI{4}{\litre}\\
		\> \> 2. Nachguss: \> \> \> \SI{2}{\litre}\\
		\> \> \> Aufhacken: \> \> \Square \ ja \> \> \Square \ nein\\
		\> \> 3. Nachguss: \> \> \> \SI{2}{\litre}\\
		\> Würze: \> \> \> \rule[-0.1cm]{0.6cm}{0.5pt}\si{\plato} \> bei \> \rule[-0.1cm]{0.6cm}{0.5pt}\si{\degreeCelsius} \> $\Rightarrow$ \> \rule[-0.1cm]{0.6cm}{0.5pt}\si{\plato}\\
		\> Glattwasser: \> \> \> \rule[-0.1cm]{0.6cm}{0.5pt}\si{\plato} \> bei \> \rule[-0.1cm]{0.6cm}{0.5pt}\si{\degreeCelsius} \> $\Rightarrow$ \> \rule[-0.1cm]{0.6cm}{0.5pt}\si{\plato}\\\\
		\> Kommentar: \>\>\> \rule[-0.1cm]{13cm}{0.5pt}\\
			\> \>  \rule[-0.1cm]{15.3cm}{0.5pt}\\
			\> \>  \rule[-0.1cm]{15.3cm}{0.5pt}\\
			\> \>  \rule[-0.1cm]{15.3cm}{0.5pt}	
	\end{tabbing}
%
%Würzekochung
\paragraph{Würzekochung:} für 70 Minuten von \Zeitluecke -- \Zeitluecke Uhr.
	\begin{tabbing}
		\hspace{1cm} \= \hspace{1cm} \= \hspace{1cm} \= \hspace{1cm} \= \hspace{1cm} \= \hspace{1cm} \= \hspace{1cm} \= \hspace{1cm} \= \kill
		\> Kommentar: \>\>\> \rule[-0.1cm]{13cm}{0.5pt}\\
			\> \>  \rule[-0.1cm]{15.3cm}{0.5pt}\\
			\> \>  \rule[-0.1cm]{15.3cm}{0.5pt}\\
			\> \>  \rule[-0.1cm]{15.3cm}{0.5pt}		
	\end{tabbing}
%\rule[-0.1cm]{1cm}{0.5pt}
%Whirpool
\paragraph{Whirpoolrast:} von \Zeitluecke -- \Zeitluecke Uhr.
	\begin{tabbing}
		\hspace{1cm} \= \hspace{1cm} \= \hspace{1cm} \= \hspace{1cm} \= \hspace{1cm} \= \hspace{1cm} \= \hspace{1cm} \= \hspace{1cm} \= \kill
		\> \> \> \rule[-0.1cm]{0.6cm}{0.5pt}\si{\plato} \> bei \> \rule[-0.1cm]{0.6cm}{0.5pt}\si{\degreeCelsius} \> $\Rightarrow$ \> \rule[-0.1cm]{0.6cm}{0.5pt}\si{\plato}\\
		\> \> Verdünnen: \> \> \> \> \> ($\rule[-0.1cm]{1cm}{0.5pt}\si{\litre} \cdot \rule[-0.1cm]{1cm}{0.5pt}\si{\plato})/(\rule[-0.1cm]{1cm}{0.5pt}\si{\plato})= \rule[-0.1cm]{1cm}{0.5pt}\si{\litre}$\\
		\> \> Sudhausausbeute: \> \> \> \> \> \rule[-0.1cm]{1cm}{0.5pt}\si{\percent} mit \SI{10}{\litre}\\
		\> \> Geschätzter Alkoholgehalt: \> \> \> \> \> \rule[-0.1cm]{1cm}{0.5pt}\si{\pervol}.\\
				\> Kommentar: \>\>\> \rule[-0.1cm]{13cm}{0.5pt}\\
		\> \>  \rule[-0.1cm]{15.3cm}{0.5pt}\\
		\> \>  \rule[-0.1cm]{15.3cm}{0.5pt}\\
		\> \>  \rule[-0.1cm]{15.3cm}{0.5pt}
	\end{tabbing}
%
%Hefegabe
\paragraph{Hefegabe:} und $O_2$-Gabe um \Zeitluecke Uhr.
%
\pagebreak
\subsection*{Gärverlauf}
	\begin{tabbing}
		\hspace{1cm} \= \hspace{1cm} \= \hspace{1cm} \= \hspace{1cm} \= \hspace{1cm} \= \hspace{1cm} \= \hspace{1cm} \= \hspace{1cm} \= \kill
		\> Angestellt am \rule[-0.1cm]{3cm}{0.5pt}, den \rule[-0.1cm]{4cm}{0.5pt} \ um \Zeitluecke Uhr.\\
		\> \> ungefähre Gärtemperatur \rule[-0.1cm]{1cm}{0.5pt}\si{\degreeCelsius}.\\
		\> Erste Kräusen am \rule[-0.1cm]{4cm}{0.5pt}, Tag Nr.\rule[-0.1cm]{1cm}{0.5pt} .\\
		\> \> ungefähre Gärtemperatur \rule[-0.1cm]{1cm}{0.5pt}\si{\degreeCelsius}.\\
		\> Kräusen fallen zusammen am \rule[-0.1cm]{4cm}{0.5pt}, Tag Nr. \rule[-0.1cm]{1cm}{0.5pt}.\\
		\> \> ungefähre Gärtemperatur \rule[-0.1cm]{1cm}{0.5pt}\si{\degreeCelsius}.\\
		\> Gärung beendet am \rule[-0.1cm]{4cm}{0.5pt}, Tag Nr. \rule[-0.1cm]{1cm}{0.5pt}.\\
		\> \> ungefähre Gärtemperatur \rule[-0.1cm]{1cm}{0.5pt}\si{\degreeCelsius}.
	\end{tabbing}
%
\subsection*{Schlauchen}
	\begin{tabbing}
		\hspace{1cm} \= \hspace{1cm} \= \hspace{1cm} \= \hspace{1cm} \= \hspace{1cm} \= \hspace{1cm} \= \hspace{1cm} \= \hspace{1cm} \= \kill
		\> \rule[-0.1cm]{4cm}{0.5pt}, den \rule[-0.1cm]{4cm}{0.5pt} um \Zeitluecke Uhr.\\
		\> Zuckergabe: \>\>\>\>\rule[-0.1cm]{1cm}{0.5pt} \>\si{\gram\per\litre}\\
		\> Gesamtabfüllung: \> \> \> \> \rule[-0.1cm]{1cm}{0.5pt}\> \SI{0,33}{\litre} \>Flasche\\
		\> \> \> \> \>\rule[-0.1cm]{1cm}{0.5pt} \> \SI{0,5}{\litre} \>Flasche\\
		\> \> \> \> \> \rule[-0.1cm]{1cm}{0.5pt}\> \SI{0,75}{\litre} \>Flasche\\
		\> Kommentar: \>\>\> \rule[-0.1cm]{13cm}{0.5pt}\\
		\> \>  \rule[-0.1cm]{15.3cm}{0.5pt}\\
		\> \>  \rule[-0.1cm]{15.3cm}{0.5pt}\\
		\> \>  \rule[-0.1cm]{15.3cm}{0.5pt}		
	\end{tabbing}
%
\subsection*{Verkosten}
\begin{tabbing}
	\hspace{1cm} \= \hspace{1cm} \= \hspace{1cm} \= \hspace{1cm} \= \hspace{1cm} \= \hspace{1cm} \= \hspace{1cm} \= \hspace{1cm} \= \kill
	\> Kommentar: \>\>\> \rule[-0.1cm]{13cm}{0.5pt}\\
	\> \>  \rule[-0.1cm]{15.3cm}{0.5pt}\\
	\> \>  \rule[-0.1cm]{15.3cm}{0.5pt}\\
	\> \>  \rule[-0.1cm]{15.3cm}{0.5pt}\\		
	\> \>  \rule[-0.1cm]{15.3cm}{0.5pt}\\
	\> \>  \rule[-0.1cm]{15.3cm}{0.5pt}\\
	\> \>  \rule[-0.1cm]{15.3cm}{0.5pt}\\
	\> \>  \rule[-0.1cm]{15.3cm}{0.5pt}\\
	\> \>  \rule[-0.1cm]{15.3cm}{0.5pt}\\
	\> \>  \rule[-0.1cm]{15.3cm}{0.5pt}\\
	\> \>  \rule[-0.1cm]{15.3cm}{0.5pt}\\
	\> \>  \rule[-0.1cm]{15.3cm}{0.5pt}\\
	\> \>  \rule[-0.1cm]{15.3cm}{0.5pt}\\
	\> \>  \rule[-0.1cm]{15.3cm}{0.5pt}\\
	\> \>  \rule[-0.1cm]{15.3cm}{0.5pt}\\
	\> \>  \rule[-0.1cm]{15.3cm}{0.5pt}
\end{tabbing}}
\end{document}
