\documentclass[12pt,oneside,a4paper]{scrartcl}
%Einstellungen der Seitenraender
\usepackage[left=1.5cm,right=1.5cm,top=2cm,bottom=2.5cm]{geometry}

\usepackage{ngerman}
\usepackage[utf8]{inputenc}
\usepackage{setspace}
\usepackage[pdftex]{graphicx}

%Zum Erstellen von Diagrammen
\usepackage{tikz}
\usepackage{pgfplots}

\usepackage{bigstrut}

%Hyperlinks innerhalb des PDF Dokuments
\usepackage[bookmarks,
ngerman,
pageanchor,
hyperindex,
hidelinks,
pdffitwindow,
pdftitle={Sudprotokoll: Dinkelbier / Industriebier},
pdfauthor={Stephan Mertens},
]{hyperref}
\usepackage{bookmark}

%um text einzufaerben
\usepackage{color}

%tabular zeug
\usepackage{dcolumn}

% Fuer zusaaetzliche Zeichen
\usepackage{textcomp}
\usepackage{wasysym}
\usepackage{marvosym}

\begin{document}
%Header
	\begin{minipage}[c]{0.70\textwidth}
		\section*{\hspace{-.4cm}Sudprotokoll: Dinkelbier / Industriebier}
	\end{minipage}
	\begin{minipage}[c]{0.29\textwidth}
		\subsection*{am 04. Juni 2016}
	\end{minipage}
	\rule{\textwidth}{1pt}
%
\subsection*{Zutaten}
%
%Schuettung
\paragraph{Schüttung:}
	\begin{tabular}[t]{m{8cm} m{2cm} m{1cm}}
		Dinkelmalz & 1,04 & kg \bigstrut\\
		Münchner Malz & 1,2 & kg \bigstrut\\
		Cara Hell & 0,24 & kg \bigstrut\\\hline
		Gesamtschüttung & 2,4 & kg\bigstrut
	\end{tabular}\\

\vspace{.25cm}
\hspace{1cm}Angestrebte Stammwürze: 13 °P
%
%Hopfung
\paragraph{Hopfung:}
	\begin{tabular}[t]{m{2.5cm} m{5cm} m{0.5cm} m{1cm} m{0.5cm} m{1cm}}
		Bitterhopen: & Magnum & 4 & g &  &12,2\%$\alpha$ \\
		Aromahopfen: & Tettnanger & 12 & g &  &3,7\%$\alpha$
	\end{tabular}\\

\vspace{.25cm}
\hspace{1cm}Errechnete Bittere: 14 IBU
%
%Hefe
\paragraph{Hefe:}
	Safbrew WB-06

\onehalfspacing{
%
%Sudverlauf
\subsection*{Sudverlauf - Infusionsverfahren}	
%
%Maischen
\paragraph{Maischen:}
	\begin{tabbing}\hspace{1cm} \=
		\hspace{1cm} \= \hspace{1cm} \=\hspace{1cm} \=\hspace{1cm} \=\hspace{1cm} \= \hspace{1cm} \= \hspace{1cm} \= \hspace{1cm} \= \hspace{1cm} \= \kill
		\> 7l vorlegen bei 55$^\circ C$.\\
		%\> \> Aufheizen auf 35$^ \circ C$.\\
		%\> \textit{Gummi/Glucanaserast} für Minuten von - Uhr.\\
		%\> \> Aufheizen auf 44$^\circ C$.\\
		%\> \textit{Ferulasäurerast} für Minuten von - Uhr.\\
		%\> \> Aufheizen auf 55$^\circ C$.\\
		\> \textit{Eiweißrast} für 15 Minuten von 8:55 - 9:13 Uhr.\\
		\> \> Aufheizen auf 64$^\circ C$\\
		\> \textit{Maltose-/$\beta$-Amylaserast} für 35 Minuten von 9:19 - 10:00 Uhr.\\
		\> \> Aufheizen auf 72$^\circ C$.\\
		\> \textit{Verzuckerungs-/$\alpha$-Amylaserast} bis zur Verzuckerung von 10:10 - 10:55 Uhr.\\
		\> \> \> Jodprobe: \> \> \Square \ positiv \> \> \CheckedBox \ negativ\\
		\> \> Aufheizen auf 78$^\circ C$.\\\\
		%
		\> Kommentar: \>\>\> Eiweißrast bei 57-58$^\circ$C, scheint so dass ich auf 76$^\circ$C überhitzt hätte,\\
		\>\>\>\> d.h. eine Aufheizrate von 1,5 $\frac{^\circ C}{min}$, vllt. über 78$^\circ$ erhitzt\\
	\end{tabbing}
%
%Läutern
\paragraph{Läutern:} von 11:05 - 12:05 Uhr.
	\begin{tabbing}
		\hspace{1cm} \= \hspace{1cm} \= \hspace{1cm} \= \hspace{1cm} \= \hspace{1cm} \=\hspace{1cm} \=\hspace{1cm} \=\hspace{1cm} \= \kill
		\> 1l Wasser vorlegen.\\
		\> \> 1. Nachguss: \> \> 7 l\\
		\> \> 2. Nachguss: \> \> 5 l\\
		\> \> \> Aufhacken: \> \> \Square \ ja \> \> \Square \ nein\\
		%\> \> 3. Nachguss: \> \> 2 l\\
		\> Würze: \> \> \> 8,3$^\circ P$ \> bei \> 71$^\circ C$ \> $\Rightarrow$ \> 15$^\circ P$\\
		\> Glattwasser: \> \> \>1,5$^\circ P$ \> bei \> 61$^\circ C$ \> $\Rightarrow$ \> 6,5$^\circ P$\\\\
		\> Kommentar: \>\>\> Hat etwas gedauert bis es gelaufen ist, anfangs recht dreckig, Treber-\\
		\>\>\>\>kuchen bei Vorderwürze sehr trocken gelaufen, 4,2l Vorderwürze mit\\
		\>\>\>\>21$^\circ$P, +5l\\
	\end{tabbing}
%
%Würzekochung
\paragraph{Würzekochung:} für 90 Minuten von 12:15 - 13:45 Uhr.\\
%Whirpool
\paragraph{Whirpoolrast:} von 13:45 - 15:05 Uhr.
	\begin{tabbing}
		\hspace{1cm} \= \hspace{1cm} \= \hspace{1cm} \= \hspace{1cm} \= \hspace{1cm} \= \hspace{1cm} \= \hspace{1cm} \= \hspace{1cm} \= \kill
		\> Kommentar: \>\>\> Mit NT, zuerst durch das \glq grobe\grq\ Sieb und dann durch den Maische fil-\\
		\>\>\>\>ter, ging recht schnell, lief klar und in einer dunklen Bernsteinfarbe, viel\\
		\>\>\>\>Heißtrub.\\
		\> \> \> ca. 15$^\circ P$ \> \> bei \> 37$^\circ C$ \> $\Rightarrow$ \> 16,5$^\circ P$\\
		\> \> Verdünnen: \> \> \> \> \> ($7,5l \cdot$$ 16.5^\circ P)/$($14^\circ P)= 9,5l$\\
		\> \> Sudhausausbeute: \> \> \> \> \>  55,4\% mit 9,5l\\
		\> \> Geschätzter Alkoholgehalt: \> \> \> \> \> 5\% Vol.
	\end{tabbing}
%
%Hefegabe
\paragraph{Hefegabe} und $O_2$-Gabe um  Uhr.
%
\subsection*{Gärverlauf}
	\begin{tabbing}
		\hspace{1cm} \= \hspace{1cm} \= \hspace{1cm} \= \hspace{1cm} \= \hspace{1cm} \= \hspace{1cm} \= \hspace{1cm} \= \hspace{1cm} \= \kill
		\> Angestellt am , den   um  Uhr.\\
		\> \> ungefähre Gärtemperatur  $^\circ C$.\\
		\> Erste Kräusen am \hspace{4cm}, Tag Nr.\hspace{2.5cm} .\\
		\> \> ungefähre Gärtemperatur \hspace{2.5cm}$^\circ C$.\\
		\> Kräusen fallen zusammen am \hspace{4cm}, Tag Nr. \hspace{2.5cm}.\\
		\> \> ungefähre Gärtemperatur \hspace{2.5cm}$^\circ C$.\\
		\> Gärung beendet am \hspace{4cm}, Tag Nr. \hspace{2.5cm}.\\
		\> \> ungefähre Gärtemperatur \hspace{2.5cm}$^\circ C$.
	\end{tabbing}
%
\subsection*{Schlauchen}
	\begin{tabbing}
		\hspace{1cm} \= \hspace{1cm} \= \hspace{1cm} \= \hspace{1cm} \= \hspace{1cm} \= \hspace{1cm} \= \hspace{1cm} \= \hspace{1cm} \= \kill
		\> \hspace{4cm}, den \hspace{4cm} um \hspace{2.5cm} Uhr.\\
		\> Zuckergabe: \hspace{3.2cm} g/l\\
		\> Gesamtabfüllung: \> \> \> \> \> 0,33l Flasche\\
		\> \> \> \> \> \> 0,5l Flasche\\
		\> \> \> \> \> \> 0,75l Flasche\\
		\> Kommentar: \>\>\> \rule[-0.2cm]{13cm}{1pt}\\
		\> \>  \rule[-0.2cm]{15.3cm}{1pt}\\
		\> \>  \rule[-0.2cm]{15.3cm}{1pt}\\
		\> \>  \rule[-0.2cm]{15.3cm}{1pt}		
	\end{tabbing}
%
\subsection*{Verkostet}
\begin{tabbing}
	\hspace{1cm} \= \hspace{1cm} \= \hspace{1cm} \= \hspace{1cm} \= \hspace{1cm} \= \hspace{1cm} \= \hspace{1cm} \= \hspace{1cm} \= \kill
	\> Kommentar: \>\>\> \rule[-0.2cm]{13cm}{1pt}\\
	\> \>  \rule[-0.2cm]{15.3cm}{1pt}\\
	\> \>  \rule[-0.2cm]{15.3cm}{1pt}\\
	\> \>  \rule[-0.2cm]{15.3cm}{1pt}\\		
	\> \>  \rule[-0.2cm]{15.3cm}{1pt}\\
	\> \>  \rule[-0.2cm]{15.3cm}{1pt}\\
	\> \>  \rule[-0.2cm]{15.3cm}{1pt}\\
	\> \>  \rule[-0.2cm]{15.3cm}{1pt}\\
	\> \>  \rule[-0.2cm]{15.3cm}{1pt}\\
	\> \>  \rule[-0.2cm]{15.3cm}{1pt}\\
	\> \>  \rule[-0.2cm]{15.3cm}{1pt}\\
	\> \>  \rule[-0.2cm]{15.3cm}{1pt}\\
	\> \>  \rule[-0.2cm]{15.3cm}{1pt}\\
	\> \>  \rule[-0.2cm]{15.3cm}{1pt}\\
	\> \>  \rule[-0.2cm]{15.3cm}{1pt}\\
	\> \>  \rule[-0.2cm]{15.3cm}{1pt}\\
	\> \>  \rule[-0.2cm]{15.3cm}{1pt}\\
	\> \>  \rule[-0.2cm]{15.3cm}{1pt}\\
	\> \>  \rule[-0.2cm]{15.3cm}{1pt}\\
	\> \>  \rule[-0.2cm]{15.3cm}{1pt}
\end{tabbing}}
\end{document}
