\documentclass[12pt,oneside,a4paper]{scrartcl}
%Einstellungen der Seitenraender
\usepackage[left=1.5cm,right=1.5cm,top=2cm,bottom=2.5cm]{geometry}

\usepackage{ngerman}
\usepackage[utf8]{inputenc}
\usepackage{setspace}
\usepackage[pdftex]{graphicx}

%Zum Erstellen von Diagrammen
\usepackage{tikz}
\usepackage{pgfplots}

\usepackage{bigstrut}
\usepackage{siunitx}
\sisetup{
	locale = DE ,
	per-mode = symbol
}

%Hyperlinks innerhalb des PDF Dokuments
\usepackage[bookmarks,
ngerman,
pageanchor,
hyperindex,
hidelinks,
pdffitwindow,
pdftitle={Sudprotokoll: Imperial Kaffee-Oatmeal-Stout},
pdfauthor={Stephan Mertens},
]{hyperref}
\usepackage{bookmark}

%um text einzufaerben
\usepackage{color}

%tabular zeug
\usepackage{dcolumn}

% Fuer zusaaetzliche Zeichen
\usepackage{textcomp}
\usepackage{wasysym}
\usepackage{marvosym}

\begin{document}
%Header
	\begin{minipage}[c]{0.70\textwidth}
		\section*{\hspace{-.4cm}Sudprotokoll: Imperial Kaffee-Oatmeal-Stout }
	\end{minipage}
	\begin{minipage}[c]{0.29\textwidth}
		\subsection*{am 19. Nov. 2016}
	\end{minipage}
	\rule{\textwidth}{1pt}
%
\subsection*{Zutaten}
%
%Schuettung
\paragraph{Schüttung:}
	\begin{tabular}[t]{m{8cm} m{2cm} m{1cm}}
		Münchner Malz & \num{2,55}  & kg \bigstrut\\
		Pale Ale Malt & \num{1,45} & kg \bigstrut\\
		Haferflocken & \num{0,3} & kg \bigstrut\\
		Sauermalz & \num{0,055} & kg \bigstrut\\ \hline
		Hauptschüttung & \num{4,355} & kg \bigstrut\\
		&&\\
		Espressopulver (Lavazza) & \num{0,1} & kg \bigstrut\\\hline
		Zweitschüttung & \num{0,1} & kg \bigstrut\\
		&&\\\hline\hline
		Gesamtschüttung & \num{4,455} & kg\bigstrut
	\end{tabular}\\

\vspace{.25cm}
\hspace{1cm}Angestrebte Stammwürze: 20\,°P
%
%Hopfung
\paragraph{Hopfung:}
	\begin{tabular}[t]{l l c r r r}
		Bitterhopfen & Magnum &  & \num{10} &  \si{\gram}& \SI{12,2}{\percent}\,$\alpha$ \\
		Aromahopfen: & Tettnanger &  & \num{30} & \si{\gram} & \SI{3,7}{\percent}\,$\alpha$
	\end{tabular}\\

\vspace{.25cm}
\hspace{1cm}Errechnete Bittere: 30\,IBU
%
%Hefe
\paragraph{Hefe:}
	Safale US-05

%\onehalfspacing{
%
%Sudverlauf
\subsection*{Sudverlauf - Infusionsverfahren}	

%Maischen
\paragraph{Maischen:}
	\begin{tabbing}\hspace{1cm} \=
		\hspace{1cm} \= \hspace{1cm} \=\hspace{1cm} \=\hspace{1cm} \=\hspace{1cm} \= \hspace{1cm} \= \hspace{1cm} \= \hspace{1cm} \= \hspace{1cm} \= \kill
		\> \SI{7,5}{\litre} vorlegen bei \SI{50}{\celsius}.\\
		%\> \> Aufheizen auf 35$^ \circ C$.\\
		%\> \textit{Gummi/Glucanaserast} für Minuten von - Uhr.\\
		%\> \> Aufheizen auf 44$^\circ C$.\\
		%\> \textit{Ferulasäurerast} für Minuten von - Uhr.\\
		%\> \> Aufheizen auf 52$^\circ C$.\\
		\> \textit{Eiweißrast} für 20 Minuten von 13:32 - 13:52 Uhr.\\
		\> \> Aufheizen auf \SI{64}{\celsius}\\
		\> \textit{Maltose-/$\beta$-Amylaserast} für 30 Minuten von 14:08 - 14:38 Uhr.\\
		\> \> Aufheizen auf \SI{72}{\celsius}.\\
		\> \textit{Verzuckerungs-/$\alpha$-Amylaserast} bis zur Verzuckerung von 14:48 - 18:00 Uhr.\\
		\> \> \> Jodprobe: \> \> \CheckedBox \ positiv \> \> \Square \ negativ\\
		\> \> Aufheizen auf \SI{78}{\celsius}.\\\\
		%
		\> Kommentar: \>\>\> Alten Weck-Einmachkocher benutzt. Temperatur mit Thermometer kon-\\
		\>\>\>trolliert. \SI{64}{\celsius} Rast auf \SI{60}{\minute} verlängern. \SI{3}{\hour} bei \SI{72}{\celsius} hat für Verzuckerung \\
		\>\>\>nicht genügt. Brauerjod, ist sehr alt, vllt. funktioniert es nicht mehr.\\
	\end{tabbing}

%Läutern
\paragraph{Läutern:} von 18:40 - 19:50 Uhr.
	\begin{tabbing}
		\hspace{1cm} \= \hspace{1cm} \= \hspace{1cm} \= \hspace{1cm} \= \hspace{1cm} \=\hspace{1cm} \=\hspace{1cm} \=\hspace{1cm} \= \kill
		\> 1l Wasser vorlegen.\\
		\> \> 1. Nachguss: \> \> \> \SI{4,5}{\litre}\\
		\> \> 2. Nachguss: \> \> \> \SI{3}{\litre}\\
		\> \> \> Aufhacken: \> \> \CheckedBox \ ja \> \> \Square \ nein\\
		\> \> 3. Nachguss: \> \> \> \SI{2}{\litre}\\
		\> Würze: \> \> \> 14\,°P\> bei \> \SI{41}{\celsius} \> $\Rightarrow$ \> \num{15,8}\,°P\\
		\> \> \> ca. \SI{11}{\litre} ausgeschlagen.\\
		\> \> \> $\Rightarrow$ bis auf \SI{9}{\litre} einkochen.\\ \\
		\> Kommentar: \>\>\>Erster Nachguss sofort dazugegeben. Schon beim Vorschießen läuft die Wür-\\
		\>\>\>ze schlecht. Früh verstopft, aufgehackt bevor der erste Liter geläutert ist. Läuft beschis-\\
		\>\>\>sen, schon zwei mal aufgehackt bevor der zweite Nachguss kommt. Vllt. auch zu fein\\
		\>\>\>geschrotet. Treberkuchen macht schnell dicht und ist schwer und zäh beim umrühren.\\
		\>\>\>Treber trockengelaufen, zweiter Nachguss mit ca. \SI{90}{\celsius} zugegeben. dann wieder aufge-\\
		\>\>\>hackt. Zweiter Nachguss läuft nach kurzer Zeit recht gut. Mit ca. \SI{7}{\litre} den Treber völlig\\
		\> \> \>trocken laufen gelassen. Ausschlagmenge reicht nicht $\Rightarrow$ dritter Nachguss. Der ist zu \\
		\> \> \>wenig um Treber zu befeuchten deswegen auf \SI{3,5}{\litre} erhöht. Dritter Nachguss läuft mit\\
		\> \> \>einer feinen Trübe. Weiterer Jodtest beim Aufheizen ist negativ. Brauerjod scheint\\
		\> \> \>doch zu funktionieren.\\
	\end{tabbing}

%Würzekochung
\paragraph{Würzekochung:} für 90 Minuten von 20:36 - 22:07 Uhr.
	\begin{tabbing}
		\hspace{1cm} \= \hspace{1cm} \= \hspace{1cm} \= \hspace{1cm} \= \hspace{1cm} \= \hspace{1cm} \= \hspace{1cm} \= \hspace{1cm} \= \kill
		\> Kommentar: \> \> \> Zu wenig Bitterhopfen mitgenommen, deswegen nur 30\,IBU statt 50\,IBU. Waa-\\
		\> \> \>ge ist aber ungenau (zählt rückwärts). Weck hat Probleme bei offenem Deckel zu Ko-\\
		\> \> \>chen. Topf ist außen sehr heiß. Immer wieder von \glqq Entsaften\grqq\ auf Kochen gestellt (Ko-\\
		\>\>\>chen kocht nicht) um kein Überlastung der Sicherung zu riskieren. In \SI{90}{\minute} kaum Vo-\\
		\>\>\>lumen verloren. Kochung auf \glqq Entsaften\grqq\ um \SI{30}{\minute} verlängert. Weck ist aus. Denk\\
		\>\>\>mal der ist zu warm geworden. Deswegen nicht weiter gekocht. Espresso in fünf Tee-\\
		\>\>\>beuteln in den Whirpool geworfen.\\
	\end{tabbing}

%Whirpool
\paragraph{Whirpoolrast:} von 22:08 - 22:48 Uhr.\\

\hspace{1cm} Zweite Schüttung (in Teebeuteln) in den vollen Whirpool gelegt.
	\begin{tabbing}
		\hspace{1cm} \= \hspace{1cm} \= \hspace{1cm} \= \hspace{1cm} \= \hspace{1cm} \= \hspace{1cm} \= \hspace{1cm} \= \hspace{1cm} \= \kill
		\> Kommentar: \>\>\>  \\
		\> \> \> ca. 16\,°P \> \> bei \> \SI{51}{\celsius} \> $\Rightarrow$ \> 19\,°P\\
		\> \> Sudhausausbeute: \> \> \> \> \>  \SI{50}{\percent} mit \SI{10}{\litre}\\
		\> \> Geschätzter Alkoholgehalt: \> \> \> \> \> \SI{8,5}{\percent}Vol.\\
		\>Kommentar: \>\>\>NT. Anstellwürze ist heller als erhofft und hat auf den ersten Eindruck kei-\\
		\>\>\>ne Espressoaromen. Ansonsten gute Trennung, viel Heißtrub. Würze schmeckt mal-\\
		\>\>\>zig, etwas dumpf, könnte von Kaffee kommen. Bittere kommt stark und kurz, fast\\
		\>\>\>etwas kräutrig.
	\end{tabbing}
%
%Hefegabe
\paragraph{Hefegabe} und $O_2$-Gabe Sonntag um 12:12 Uhr. 
%
\subsection*{Gärverlauf}
	\begin{tabbing}
		\hspace{1cm} \= \hspace{1cm} \= \hspace{1cm} \= \hspace{1cm} \= \hspace{1cm} \= \hspace{1cm} \= \hspace{1cm} \= \hspace{1cm} \= \kill
		\> Angestellt am Sonntag, den  20.11. um 12:12 Uhr.\\
		\> \> ungefähre Gärtemperatur \SI{18}{\celsius}.\\
		\> Erste Kräusen am Montag, Tag Nr.1.\\
		\> \> ungefähre Gärtemperatur \SI{18}{\celsius}.\\
		\> Gärung beendet am Samstag, Tag Nr. 6.\\
		\> \> ungefähre Gärtemperatur \SI{17}{\celsius}.
	\end{tabbing}
%
\subsection*{Schlauchen}
	\begin{tabbing}
		\hspace{1cm} \= \hspace{1cm} \= \hspace{1cm} \= \hspace{1cm} \= \hspace{1cm} \= \hspace{1cm} \= \hspace{1cm} \= \hspace{1cm} \= \kill
		\> Samstag, den 03.12. um 19:00 Uhr.\\
		\> Zuckergabe: \SI{3,3}{\gram\per\litre}\\
		\> Gesamtabfüllung: \> \> \> \> 29\> \SI{0,33}{\litre} Flaschen\\
		\> \> \> \> \> 2\> \SI{0,5}{\litre} Flaschen\\
		\> Kommentar: \>\>\> Trennung lief mit einer feinen Trübung. Locken schwammen oben auf,\\
		\>\>vllt. nächstes mal Aufkräusen. Bier über zwei Töpfe gepuffert.
	\end{tabbing}
%
\subsection*{Verkostet}
\begin{tabbing}
	\hspace{1cm} \= \hspace{1cm} \= \hspace{1cm} \= \hspace{1cm} \= \hspace{1cm} \= \hspace{1cm} \= \hspace{1cm} \= \hspace{1cm} \= \kill
	\> Kommentar: \>\>\> Probieren nach dem Schlauchen. Optisch hellbraun mit feiner Trübung.\\
	\>\>Riecht etwas nach Karamell und etwas würzigem Hopfen. Im Hintergrund ist noch \\
	\>\>was anderes. Geschmacklich ist es anfangs samtig, dann stark bitter mit einem leicht \\
	\>\>alkoholischen Abgang. Dazu kommt eine würzige Kopfnote und wenn man drauf ach-\\
	\>\>tet eine minimale Kaffeenote. Idee fürs nächste Mal: Anstellwürze vom Whirpool \\
	\>\>durch einen Melittafilter in den Gäreimer füllen (auf Heißoxidation achten).\\
	\>\>\>\>Probieren am 
\end{tabbing}
\end{document}
