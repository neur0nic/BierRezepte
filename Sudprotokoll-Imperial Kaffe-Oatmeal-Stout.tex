\documentclass[12pt,oneside,a4paper]{scrartcl}
%Einstellungen der Seitenraender
\usepackage[left=1.5cm,right=1.5cm,top=2cm,bottom=2.5cm]{geometry}

\usepackage{ngerman}
\usepackage[utf8]{inputenc}
\usepackage{setspace}
\usepackage[pdftex]{graphicx}

%Zum Erstellen von Diagrammen
\usepackage{tikz}
\usepackage{pgfplots}

\usepackage{bigstrut}
\usepackage{siunitx}

%Hyperlinks innerhalb des PDF Dokuments
\usepackage[bookmarks,
ngerman,
pageanchor,
hyperindex,
hidelinks,
pdffitwindow,
pdftitle={Sudprotokoll: Imperial Kaffee-Oatmeal-Stout},
pdfauthor={Stephan Mertens},
]{hyperref}
\usepackage{bookmark}

%um text einzufaerben
\usepackage{color}

%tabular zeug
\usepackage{dcolumn}

% Fuer zusaaetzliche Zeichen
\usepackage{textcomp}
\usepackage{wasysym}
\usepackage{marvosym}

\begin{document}
%Header
	\begin{minipage}[c]{0.70\textwidth}
		\section*{\hspace{-.4cm}Sudprotokoll: Imperial Kaffee-Oatmeal-Stout }
	\end{minipage}
	\begin{minipage}[c]{0.29\textwidth}
		\subsection*{am 19. Nov. 2016}
	\end{minipage}
	\rule{\textwidth}{1pt}
%
\subsection*{Zutaten}
%
%Schuettung
\paragraph{Schüttung:}
	\begin{tabular}[t]{m{8cm} m{2cm} m{1cm}}
		Münchner Malz & \num{2,55}  & kg \bigstrut\\
		Paleale Malt & \num{1,45} & kg \bigstrut\\
		Haferflocken & \num{0,3} & kg \bigstrut\\
		Sauermalz & \num{0,055} & kg \bigstrut\\ \hline
		Hauptschüttung & \num{4,355} & kg \bigstrut\\
		&&\\
		Espressopulver & \num{0,1} & kg \bigstrut\\\hline
		Zweitschüttung & \num{0,1} & kg \bigstrut\\
		&&\\\hline\hline
		Gesamtschüttung & \num{4,455} & kg\bigstrut
	\end{tabular}\\

\vspace{.25cm}
\hspace{1cm}Angestrebte Stammwürze: 20\,ggrraaddP
%
%Hopfung
\paragraph{Hopfung:}
	\begin{tabular}{l l c c c c}
		Bitterhopfen & Magnum &  &  &  & \% $\alpha$ \\
		Aromahopfen: & Tettnanger &  &  &  &  \% $\alpha$
	\end{tabular}\\

\vspace{.25cm}
\hspace{1cm}Errechnete Bittere: 60 IBU
%
%Hefe
\paragraph{Hefe:}
	Safbrew T-58

%\onehalfspacing{
%
%Sudverlauf
\subsection*{Sudverlauf - Infusionsverfahren}	

%Maischen
\paragraph{Maischen:}
	\begin{tabbing}\hspace{1cm} \=
		\hspace{1cm} \= \hspace{1cm} \=\hspace{1cm} \=\hspace{1cm} \=\hspace{1cm} \= \hspace{1cm} \= \hspace{1cm} \= \hspace{1cm} \= \hspace{1cm} \= \kill
		\> \SI{7}{\liter} vorlegen bei \SI{50}{\celsius}.\\
		%\> \> Aufheizen auf 35$^ \circ C$.\\
		%\> \textit{Gummi/Glucanaserast} für Minuten von - Uhr.\\
		%\> \> Aufheizen auf 44$^\circ C$.\\
		%\> \textit{Ferulasäurerast} für Minuten von - Uhr.\\
		%\> \> Aufheizen auf 52$^\circ C$.\\
		\> \textit{Eiweißrast} für 20 Minuten von -  Uhr.\\
		\> \> Aufheizen auf \SI{64}{\celsius}\\
		\> \textit{Maltose-/$\beta$-Amylaserast} für 30 Minuten von - Uhr.\\
		\> \> Aufheizen auf \SI{72}{\celsius}.\\
		\> \textit{Verzuckerungs-/$\alpha$-Amylaserast} bis zur Verzuckerung von - Uhr.\\
		\> \> \> Jodprobe: \> \> \Square \ positiv \> \> \Square \ negativ\\
		\> \> Aufheizen auf \SI{78}{\celsius}.\\\\
		%
		\> Kommentar: \>\>\> \\
	\end{tabbing}

%Läutern
\paragraph{Läutern:} von - Uhr.
	\begin{tabbing}
		\hspace{1cm} \= \hspace{1cm} \= \hspace{1cm} \= \hspace{1cm} \= \hspace{1cm} \=\hspace{1cm} \=\hspace{1cm} \=\hspace{1cm} \= \kill
		\> 1l Wasser vorlegen.\\
		\> \> 1. Nachguss: \> \> 4 l\\
		\> \> 2. Nachguss: \> \> 3 l\\
		\> \> \> Aufhacken: \> \> \Square \ ja \> \> \Square \ nein\\
		\> \> 3. Nachguss: \> \> 2 l\\
		\> Würze: \> \> \> \,ggrraaddP\> bei \> \SI{666}{\celsius} \> $\Rightarrow$ \> \,ggrraaddP\\
		\> Glattwasser: \> \> \> \,ggrraaddP \> bei \> \SI{666}{\celsius} \> $\Rightarrow$ \> \,ggrraaddP\\\\
		\> Kommentar: \>\>\> \\
	\end{tabbing}

%Würzekochung
\paragraph{Würzekochung:} für 90 Minuten von - Uhr.
	\begin{tabbing}
		\hspace{1cm} \= \hspace{1cm} \= \hspace{1cm} \= \hspace{1cm} \= \hspace{1cm} \= \hspace{1cm} \= \hspace{1cm} \= \hspace{1cm} \= \kill
		\> Kommentar: \> \> \> \\
	\end{tabbing}

%Whirpool
\paragraph{Whirpoolrast:} von - Uhr.
	\begin{tabbing}
		\hspace{1cm} \= \hspace{1cm} \= \hspace{1cm} \= \hspace{1cm} \= \hspace{1cm} \= \hspace{1cm} \= \hspace{1cm} \= \hspace{1cm} \= \kill
		\> Kommentar: \>\>\>  \\
		\> \> \> ca. \,ggrraaddP \> \> bei \> \SI{666}{\celsius} \> $\Rightarrow$ \> \,ggrraaddP\\
		\> \> Verdünnen: \> \> \> \> \> $(\SI{666}{\liter} \cdot \,^\circ P)/(\,^\circ P)= \SI{666}{\liter}$\\
		\> \> Sudhausausbeute: \> \> \> \> \>  40\% mit 10l\\
		\> \> Geschätzter Alkoholgehalt: \> \> \> \> \> 9,5\% Vol.
	\end{tabbing}
%
%Hefegabe
\paragraph{Hefegabe} und $O_2$-Gabe um 6:25 Uhr.
%
\subsection*{Gärverlauf}
	\begin{tabbing}
		\hspace{1cm} \= \hspace{1cm} \= \hspace{1cm} \= \hspace{1cm} \= \hspace{1cm} \= \hspace{1cm} \= \hspace{1cm} \= \hspace{1cm} \= \kill
		\> Angestellt am , den  \ um  Uhr.\\
		\> \> ungefähre Gärtemperatur \SI{666}{\celsius}.\\
		\> Erste Kräusen am \hspace{4cm}, Tag Nr.\hspace{2.5cm} .\\
		\> \> ungefähre Gärtemperatur \SI{666}{\celsius}.\\
		\> Kräusen fallen zusammen am \hspace{4cm}, Tag Nr. \hspace{2.5cm}.\\
		\> \> ungefähre Gärtemperatur \SI{666}{\celsius}.\\
		\> Gärung beendet am \hspace{4cm}, Tag Nr. \hspace{2.5cm}.\\
		\> \> ungefähre Gärtemperatur \SI{666}{\celsius}.
	\end{tabbing}
%
\subsection*{Schlauchen}
	\begin{tabbing}
		\hspace{1cm} \= \hspace{1cm} \= \hspace{1cm} \= \hspace{1cm} \= \hspace{1cm} \= \hspace{1cm} \= \hspace{1cm} \= \hspace{1cm} \= \kill
		\> \hspace{4cm}, den \hspace{4cm} um \hspace{2.5cm} Uhr.\\
		\> Zuckergabe: \hspace{3.2cm} g/l\\
		\> Gesamtabfüllung: \> \> \> \> \> 0,33l Flasche\\
		\> \> \> \> \> \> 0,5l Flasche\\
		\> \> \> \> \> \> 0,75l Flasche\\
		\> Kommentar: \>\>\> \rule[-0.2cm]{13cm}{1pt}\\
		\> \>  \rule[-0.2cm]{15.3cm}{1pt}\\
		\> \>  \rule[-0.2cm]{15.3cm}{1pt}\\
		\> \>  \rule[-0.2cm]{15.3cm}{1pt}		
	\end{tabbing}
%
\subsection*{Verkostet}
\begin{tabbing}
	\hspace{1cm} \= \hspace{1cm} \= \hspace{1cm} \= \hspace{1cm} \= \hspace{1cm} \= \hspace{1cm} \= \hspace{1cm} \= \hspace{1cm} \= \kill
	\> Kommentar: \>\>\> \rule[-0.2cm]{13cm}{1pt}\\
	\> \>  \rule[-0.2cm]{15.3cm}{1pt}\\
	\> \>  \rule[-0.2cm]{15.3cm}{1pt}\\
	\> \>  \rule[-0.2cm]{15.3cm}{1pt}\\		
	\> \>  \rule[-0.2cm]{15.3cm}{1pt}\\
	\> \>  \rule[-0.2cm]{15.3cm}{1pt}\\
	\> \>  \rule[-0.2cm]{15.3cm}{1pt}\\
	\> \>  \rule[-0.2cm]{15.3cm}{1pt}\\
	\> \>  \rule[-0.2cm]{15.3cm}{1pt}\\
	\> \>  \rule[-0.2cm]{15.3cm}{1pt}\\
	\> \>  \rule[-0.2cm]{15.3cm}{1pt}\\
	\> \>  \rule[-0.2cm]{15.3cm}{1pt}\\
	\> \>  \rule[-0.2cm]{15.3cm}{1pt}\\
	\> \>  \rule[-0.2cm]{15.3cm}{1pt}\\
	\> \>  \rule[-0.2cm]{15.3cm}{1pt}\\
	\> \>  \rule[-0.2cm]{15.3cm}{1pt}\\
	\> \>  \rule[-0.2cm]{15.3cm}{1pt}\\
	\> \>  \rule[-0.2cm]{15.3cm}{1pt}\\
	\> \>  \rule[-0.2cm]{15.3cm}{1pt}\\
	\> \>  \rule[-0.2cm]{15.3cm}{1pt}
\end{tabbing}
\end{document}
