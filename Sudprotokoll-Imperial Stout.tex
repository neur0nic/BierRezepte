\documentclass[12pt,oneside,a4paper]{scrartcl}
%Einstellungen der Seitenraender
\usepackage[left=1.5cm,right=1.5cm,top=2cm,bottom=2.5cm]{geometry}

\usepackage{ngerman}
\usepackage[utf8]{inputenc}
\usepackage{setspace}
\usepackage[pdftex]{graphicx}

%Zum Erstellen von Diagrammen
\usepackage{tikz}
\usepackage{pgfplots}

\usepackage{bigstrut}

%Hyperlinks innerhalb des PDF Dokuments
\usepackage[bookmarks,
ngerman,
pageanchor,
hyperindex,
hidelinks,
pdffitwindow,
pdftitle={Sudprotokoll: Ingwer-Ale / Waffelmania},
pdfauthor={Stephan Mertens},
]{hyperref}
\usepackage{bookmark}

%um text einzufaerben
\usepackage{color}

%tabular zeug
\usepackage{dcolumn}

% Fuer zusaaetzliche Zeichen
\usepackage{textcomp}
\usepackage{wasysym}
\usepackage{marvosym}

\begin{document}
%Header
	\begin{minipage}[c]{0.70\textwidth}
		\section*{\hspace{-.4cm}Sudprotokoll: Imperial Stout / }
	\end{minipage}
	\begin{minipage}[c]{0.29\textwidth}
		\subsection*{am \today}
	\end{minipage}
	\rule{\textwidth}{1pt}
%
\subsection*{Zutaten}
%
%Schuettung
\paragraph{Schüttung:}
	\begin{tabular}[t]{m{8cm} m{2cm} m{1cm}}
		Pilsener Malz & 3,6 & kg \bigstrut\\
		Münchner Malz & 1 & kg \bigstrut\\
		Cara Münch & 0,2 & kg \bigstrut\\
		Haferflocken & 0,1 & kg \bigstrut\\ \hline
		Hauptschüttung & 5 & kg \bigstrut\\
		&&\\
		Röstmalz & 0,4 & kg \bigstrut\\\hline
		Zweitschüttung & 0,4 & kg \bigstrut\\\hline\hline
		&&\\
		Gesamtschüttung & 5,4 & kg\bigstrut
	\end{tabular}\\

\vspace{.25cm}
\hspace{1cm}Angestrebte Stammwürze: $>$20 °P
%
%Hopfung
\paragraph{Hopfung:}
	\begin{tabular}[t]{m{2.5cm} m{5cm} m{0.5cm} m{1cm} m{0.5cm} m{1cm}}
		Bitterhopen: & Cascade & 30 & g & 8,5 & \% $\alpha$ \\
		Aromahopfen: & Citra & 30 & g & 13,1 &  \% $\alpha$
	\end{tabular}\\

\vspace{.25cm}
\hspace{1cm}Errechnete Bittere: 60 IBU
%
%Hefe
\paragraph{Hefe:}
	Safbrew T-05
%
%Sudverlauf
\subsection*{Sudverlauf - Infusionsverfahren}	
%
%Maischen
\paragraph{Maischen:}
	\begin{tabbing}\hspace{1cm} \=
		\hspace{1cm} \= \hspace{1cm} \=\hspace{1cm} \=\hspace{1cm} \=\hspace{1cm} \= \hspace{1cm} \= \hspace{1cm} \= \hspace{1cm} \= \hspace{1cm} \= \kill
		\> 8l vorlegen bei 50$^\circ C$.\\
		%\> \> Aufheizen auf 35$^ \circ C$.\\
		%\> \textit{Gummi/Glucanaserast} für Minuten von - Uhr.\\
		%\> \> Aufheizen auf 44$^\circ C$.\\
		%\> \textit{Ferulasäurerast} für Minuten von - Uhr.\\
		%\> \> Aufheizen auf 52$^\circ C$.\\
		\> \textit{Eiweißrast} für 20 Minuten von - Uhr.\\
		\> \> Aufheizen auf 64$^\circ C$\\
		\> \textit{Maltose-/$\beta$-Amylaserast} für 30 Minuten von - Uhr.\\
		\> \> Aufheizen auf 72$^\circ C$.\\
		\> \textit{Verzuckerungs-/$\alpha$-Amylaserast} bis zur Verzuckerung von - Uhr.\\
		\> \> Zugabe der zweiten Schüttung.\\
		\> \> \> Jodprobe: \> \> \Square positiv \> \> \Square negativ\\
		\> \> Aufheizen auf 78$^\circ C$.\\
		\> \textit{Inaktivierungsrast} für 10 Minuten von - Uhr.\\\\
		\> Kommentar: \>\>\>
	\end{tabbing}
%
%Läutern
\paragraph{Läutern:} von - Uhr.
	\begin{tabbing}
		\hspace{1cm} \= \hspace{1cm} \= \hspace{1cm} \= \hspace{1cm} \= \hspace{1cm} \=\hspace{1cm} \=\hspace{1cm} \=\hspace{1cm} \= \kill
		\> 1l Wasser vorlegen.\\
		\> \> 1. Nachguss: \> \> 4 l\\
		\> \> 2. Nachguss: \> \> 3 l\\
		\> \> \> Aufhacken: \> \> \Square ja \> \> \Square nein\\\\
		\> \> 3. Nachguss: \> \> 2 l\\
		\> Würze: \> \> \> $^\circ P$ \> bei \> $^\circ C$ \> $\Rightarrow$ \> $^\circ P$\\
		\> Glattwasser: \> \> \> $^\circ P$ \> bei \> $^\circ C$ \> $\Rightarrow$ \> $^\circ P$\\\\
		\> Kommentar: \> \> \>
	\end{tabbing}
%
%Würzekochung
\paragraph{Würzekochung:} für 90 Minuten von - Uhr.
	\begin{tabbing}
		\hspace{1cm} \= \hspace{1cm} \= \hspace{1cm} \= \hspace{1cm} \= \hspace{1cm} \= \hspace{1cm} \= \hspace{1cm} \= \hspace{1cm} \= \kill
		\\
		\> Kommentar: \> \> \>
	\end{tabbing}
%
%Whirpool
\paragraph{Whirpoolrast:} von - Uhr.
	\begin{tabbing}
		\hspace{1cm} \= \hspace{1cm} \= \hspace{1cm} \= \hspace{1cm} \= \hspace{1cm} \= \hspace{1cm} \= \hspace{1cm} \= \hspace{1cm} \= \kill
		\> Kommentar: \> \> \> \\
		\> \> $\Rightarrow$ $^\circ P$ \> \> bei \> $^\circ C$ \> $\Rightarrow$ \> $^\circ P$\\
		\> \> Verdünnen: \> \> \> \> \> $(l \cdot ^\circ P)/(^\circ P)=l$\\
		\> \> Sudhausausbeute: \> \> \> \> \>  \% mit l\\
		\> \> Geschätzter Alkoholgehalt: \> \> \> \> \> \% Vol.
	\end{tabbing}
%
%Hefegabe
\paragraph{Hefegabe} und $O_2$-Gabe um Uhr.
%
\subsection*{Gärverlauf}
	\begin{tabbing}
		\hspace{1cm} \= \hspace{1cm} \= \hspace{1cm} \= \hspace{1cm} \= \hspace{1cm} \= \hspace{1cm} \= \hspace{1cm} \= \hspace{1cm} \= \kill
		\> Angestellt am Montag, den \today \ um Uhr.\\
		\> \> ungefähre Gärtemperatur $^\circ C$.\\
		\> Erste Kräusen am , Tag Nr. .\\
		\> \> ungefähre Gärtemperatur $^\circ C$.\\
		\> Kräusen fallen zusammen am , Tag Nr. .\\
		\> \> ungefähre Gärtemperatur $^\circ C$.\\
		\> Gärung beendet am , Tag Nr. .\\
		\> \> ungefähre Gärtemperatur $^\circ C$.
	\end{tabbing}
%
\subsection*{Schlauchen}
	\begin{tabbing}
		\hspace{1cm} \= \hspace{1cm} \= \hspace{1cm} \= \hspace{1cm} \= \hspace{1cm} \= \hspace{1cm} \= \hspace{1cm} \= \hspace{1cm} \= \kill
		\> , den um Uhr.\\
		\> Zuckergabe: g/l\\
		\> Gesamtabfüllung: \> \> \> 0,33l Flasche\\
		\> \> \> \> 0,5l Flasche\\
		\> \> \> \> 0,75l Flasche\\
		\> \> Kommentar: \> \>
		
	\end{tabbing}
\end{document}
