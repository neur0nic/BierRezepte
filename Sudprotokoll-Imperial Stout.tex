\documentclass[12pt,oneside,a4paper]{scrartcl}
%Einstellungen der Seitenraender
\usepackage[left=1.5cm,right=1.5cm,top=2cm,bottom=2.5cm]{geometry}

\usepackage{ngerman}
\usepackage[utf8]{inputenc}
\usepackage{setspace}
\usepackage[pdftex]{graphicx}

%Zum Erstellen von Diagrammen
\usepackage{tikz}
\usepackage{pgfplots}

\usepackage{bigstrut}

%Hyperlinks innerhalb des PDF Dokuments
\usepackage[bookmarks,
ngerman,
pageanchor,
hyperindex,
hidelinks,
pdffitwindow,
pdftitle={Sudprotokoll: Imperial Stout/Too Old For That S**t},
pdfauthor={Stephan Mertens},
]{hyperref}
\usepackage{bookmark}

%um text einzufaerben
\usepackage{color}

%tabular zeug
\usepackage{dcolumn}

% Fuer zusaaetzliche Zeichen
\usepackage{textcomp}
\usepackage{wasysym}
\usepackage{marvosym}

\begin{document}
%Header
	\begin{minipage}[c]{0.70\textwidth}
		\section*{\hspace{-.4cm}Sudprotokoll: Imperial Stout/{\large too old for that S**t} }
	\end{minipage}
	\begin{minipage}[c]{0.29\textwidth}
		\subsection*{am \today}
	\end{minipage}
	\rule{\textwidth}{1pt}
%
\subsection*{Zutaten}
%
%Schuettung
\paragraph{Schüttung:}
	\begin{tabular}[t]{m{8cm} m{2cm} m{1cm}}
		Pilsener Malz & 3,6 & kg \bigstrut\\
		Münchner Malz & 1 & kg \bigstrut\\
		Cara Münch & 0,3 & kg \bigstrut\\
		Haferflocken & 0,08 & kg \bigstrut\\ \hline
		Hauptschüttung & 5 & kg \bigstrut\\
		&&\\
		Röstmalz & 0,25 & kg \bigstrut\\\hline
		Zweitschüttung & 0,25 & kg \bigstrut\\
		&&\\\hline\hline
		Gesamtschüttung & 5,25 & kg\bigstrut
	\end{tabular}\\

\vspace{.25cm}
\hspace{1cm}Angestrebte Stammwürze: $>$20 °P
%
%Hopfung
\paragraph{Hopfung:}
	\begin{tabular}[t]{m{2.5cm} m{5cm} m{0.5cm} m{1cm} m{0.5cm} m{1cm}}
		Bitterhopen: & Cascade & 30 & g & 8,5 & \% $\alpha$ \\
		Aromahopfen: & Citra & 30 & g & 13,1 &  \% $\alpha$
	\end{tabular}\\

\vspace{.25cm}
\hspace{1cm}Errechnete Bittere: 60 IBU
%
%Hefe
\paragraph{Hefe:}
	Safbrew T-58

\onehalfspacing{
%
%Sudverlauf
\subsection*{Sudverlauf - Infusionsverfahren}	
%
%Maischen
\paragraph{Maischen:}
	\begin{tabbing}\hspace{1cm} \=
		\hspace{1cm} \= \hspace{1cm} \=\hspace{1cm} \=\hspace{1cm} \=\hspace{1cm} \= \hspace{1cm} \= \hspace{1cm} \= \hspace{1cm} \= \hspace{1cm} \= \kill
		\> 6,5l vorlegen bei 50$^\circ C$.\\
		%\> \> Aufheizen auf 35$^ \circ C$.\\
		%\> \textit{Gummi/Glucanaserast} für Minuten von - Uhr.\\
		%\> \> Aufheizen auf 44$^\circ C$.\\
		%\> \textit{Ferulasäurerast} für Minuten von - Uhr.\\
		%\> \> Aufheizen auf 52$^\circ C$.\\
		\> \textit{Eiweißrast} für 20 Minuten von 13:00 - 13:30 Uhr.\\
		\> \> Aufheizen auf 64$^\circ C$\\
		\> \textit{Maltose-/$\beta$-Amylaserast} für 30 Minuten von 13:45 - 14:15 Uhr.\\
		\> \> Aufheizen auf 72$^\circ C$.\\
		\> \textit{Verzuckerungs-/$\alpha$-Amylaserast} bis zur Verzuckerung von 14:40 - 20:10 Uhr.\\
		\> \> Zugabe der zweiten Schüttung.\\
		\> \> \> Jodprobe: \> \> \CheckedBox \ positiv \> \> \Square \ negativ\\
		\> \> Aufheizen auf 78$^\circ C$.\\\\
		%
		\> Kommentar: \>\>\> Totale Müslimaische. Eingemaischt mit Quirl $\Rightarrow$ viel Sauerstoffeintrag,\\
		\> \> ca. 0,5l kochendes Wasser hinterher gekippt. Ähndelt jetzt einer Flüssigkeit. Rast 1\\ 
		\> \> direkt überhitzt auf etw 55$^\circ C$. Temperaturmessung sehr schwer, da in der Müslimai-\\
		\> \> sche stark geschichtete Temperaturverteilung. (17:30) Thermometer war unten mit \\
		\> \> Malz verstopft, deswegen kann nicht gesagt werden ob zwischenzeitlich überhitzt wur-\\
		\> \> de, Maische schmeckt aber sehr süß. Zwischenzeitlich wieder etwas Wasser nachge-\\
		\> \> kippt. Keine Zeit mehr zu warten.
	\end{tabbing}
%
%Läutern
\paragraph{Läutern:} von 20:15 - 22:30 Uhr.
	\begin{tabbing}
		\hspace{1cm} \= \hspace{1cm} \= \hspace{1cm} \= \hspace{1cm} \= \hspace{1cm} \=\hspace{1cm} \=\hspace{1cm} \=\hspace{1cm} \= \kill
		\> 1l Wasser vorlegen.\\
		\> \> 1. Nachguss: \> \> 4 l\\
		\> \> 2. Nachguss: \> \> 3 l\\
		\> \> \> Aufhacken: \> \> \Square \ ja \> \> \Square \ nein\\
		\> \> 3. Nachguss: \> \> 2 l\\
		\> Würze: \> \> \> 17$^\circ P$ \> bei \> 67$^\circ C$ \> $\Rightarrow$ \> 22$^\circ P$\\
		\> Glattwasser: \> \> \> 10$^\circ P$ \> bei \> 56$^\circ C$ \> $\Rightarrow$ \> 14$^\circ P$\\\\
		\> Kommentar: \>\>\> Zum Abmaischen direkt den ersten Nachguss gegeben und umgerührt.\\
		\> \> Dichte feine Trübung, Treberkuchen nach 1.5l dicht $\Rightarrow$ aufhacken. Sieb verstopft, zu\\
		\> \> fein geschrotet. Etwa 2.5l später ein weiteres mal aufhacken. Kurz vor Ende musste \\
		\> \> nochmals aufgehackt werden. Viel \glqq gutes\grqq\ Glattwasser, deswegen wird ein zweiter \\
		\> \> Topf gefüllt. ca. 12l Kochwürze.
	\end{tabbing}
%
%Würzekochung
\paragraph{Würzekochung:} für 90 Minuten von 22:40 - Uhr.
	\begin{tabbing}
		\hspace{1cm} \= \hspace{1cm} \= \hspace{1cm} \= \hspace{1cm} \= \hspace{1cm} \= \hspace{1cm} \= \hspace{1cm} \= \hspace{1cm} \= \kill
		\> Kommentar: \> \> \> Zum Schluss nur noch ein Topf gehabt (zusammengekippt)
	\end{tabbing}
%
%Whirpool
\paragraph{Whirpoolrast:} von 0:15 - 0:30 Uhr.
	\begin{tabbing}
		\hspace{1cm} \= \hspace{1cm} \= \hspace{1cm} \= \hspace{1cm} \= \hspace{1cm} \= \hspace{1cm} \= \hspace{1cm} \= \hspace{1cm} \= \kill
		\> Kommentar: \>\>\> NT. Trubkegel sehr groß, evtl. längere Rast.\\
		\> \> \> ca. 22$^\circ P$ \> \> bei \> 62$^\circ C$ \> $\Rightarrow$ \> 26$^\circ P$\\
		\> \> Verdünnen: \> \> \> \> \> (8,5$l \cdot$26$ ^\circ P)/$(22$^\circ P)=$ 10$l$\\
		\> \> Sudhausausbeute: \> \> \> \> \>  40\% mit 10l\\
		\> \> Geschätzter Alkoholgehalt: \> \> \> \> \> 9,5\% Vol.
	\end{tabbing}
%
%Hefegabe
\paragraph{Hefegabe} und $O_2$-Gabe um \hspace{2.5cm} Uhr.
%
\subsection*{Gärverlauf}
	\begin{tabbing}
		\hspace{1cm} \= \hspace{1cm} \= \hspace{1cm} \= \hspace{1cm} \= \hspace{1cm} \= \hspace{1cm} \= \hspace{1cm} \= \hspace{1cm} \= \kill
		\> Angestellt am Montag, den \today \ um \hspace{2.5cm} Uhr.\\
		\> \> ungefähre Gärtemperatur \hspace{2.5cm} $^\circ C$.\\
		\> Erste Kräusen am \hspace{4cm}, Tag Nr.\hspace{2.5cm} .\\
		\> \> ungefähre Gärtemperatur \hspace{2.5cm}$^\circ C$.\\
		\> Kräusen fallen zusammen am \hspace{4cm}, Tag Nr. \hspace{2.5cm}.\\
		\> \> ungefähre Gärtemperatur \hspace{2.5cm}$^\circ C$.\\
		\> Gärung beendet am \hspace{4cm}, Tag Nr. \hspace{2.5cm}.\\
		\> \> ungefähre Gärtemperatur \hspace{2.5cm}$^\circ C$.
	\end{tabbing}
%
\subsection*{Schlauchen}
	\begin{tabbing}
		\hspace{1cm} \= \hspace{1cm} \= \hspace{1cm} \= \hspace{1cm} \= \hspace{1cm} \= \hspace{1cm} \= \hspace{1cm} \= \hspace{1cm} \= \kill
		\> \hspace{4cm}, den \hspace{4cm} um \hspace{2.5cm} Uhr.\\
		\> Zuckergabe: \hspace{2.5cm} g/l\\
		\> Gesamtabfüllung: \> \> \> \> 0,33l Flasche\\
		\> \> \> \> \> 0,5l Flasche\\
		\> \> \> \> \> 0,75l Flasche\\
		\> Kommentar: \>\>\> \rule[-0.2cm]{13cm}{1pt}\\
		\> \>  \rule[-0.2cm]{15.3cm}{1pt}\\
		\> \>  \rule[-0.2cm]{15.3cm}{1pt}\\
		\> \>  \rule[-0.2cm]{15.3cm}{1pt}		
	\end{tabbing}
%
\subsection*{Verkostet}
\begin{tabbing}
	\hspace{1cm} \= \hspace{1cm} \= \hspace{1cm} \= \hspace{1cm} \= \hspace{1cm} \= \hspace{1cm} \= \hspace{1cm} \= \hspace{1cm} \= \kill
	\> \hspace{4cm}, den \hspace{4cm} um \hspace{2.5cm} Uhr.\\
	\> Kommentar: \>\>\> \rule[-0.2cm]{13cm}{1pt}\\
	\> \>  \rule[-0.2cm]{15.3cm}{1pt}\\
	\> \>  \rule[-0.2cm]{15.3cm}{1pt}\\
	\> \>  \rule[-0.2cm]{15.3cm}{1pt}\\		
	\> \>  \rule[-0.2cm]{15.3cm}{1pt}\\
	\> \>  \rule[-0.2cm]{15.3cm}{1pt}\\
	\> \>  \rule[-0.2cm]{15.3cm}{1pt}\\
	\> \>  \rule[-0.2cm]{15.3cm}{1pt}\\
	\> \>  \rule[-0.2cm]{15.3cm}{1pt}\\
	\> \>  \rule[-0.2cm]{15.3cm}{1pt}\\
	\> \>  \rule[-0.2cm]{15.3cm}{1pt}\\
	\> \>  \rule[-0.2cm]{15.3cm}{1pt}\\
	\> \>  \rule[-0.2cm]{15.3cm}{1pt}\\
	\> \>  \rule[-0.2cm]{15.3cm}{1pt}\\
	\> \>  \rule[-0.2cm]{15.3cm}{1pt}\\
	\> \>  \rule[-0.2cm]{15.3cm}{1pt}\\
	\> \>  \rule[-0.2cm]{15.3cm}{1pt}\\
	\> \>  \rule[-0.2cm]{15.3cm}{1pt}\\
	\> \>  \rule[-0.2cm]{15.3cm}{1pt}\\
	\> \>  \rule[-0.2cm]{15.3cm}{1pt}
\end{tabbing}}
\end{document}
